
\documentclass[12pt,a4paper]{report}
\usepackage[utf8]{inputenc}
%\usepackage[frenchb]{babel}
\usepackage[T1]{fontenc}
\usepackage{amsmath}
\usepackage{amsfonts}
\usepackage{amssymb}
\usepackage{graphicx}
\usepackage{float} % pour imposer la position avec H
\usepackage{hyperref}
\usepackage{listings} % pour insérer des listings
\usepackage[lofdepth,lotdepth]{subfig} % pôur les sous-figures
\usepackage[french]{isodate}
\usepackage{lmodern}
\usepackage{textcomp}
\usepackage[margin=2cm]{geometry}
\usepackage{tabularx}
\usepackage{hyperref}

\usepackage{xcolor}



\renewcommand{\baselinestretch}{1.00} 


\definecolor{light-gray}{gray}{0.9}

% Default fixed font does not support bold face
\DeclareFixedFont{\ttb}{T1}{txtt}{bx}{n}{12} % for bold
\DeclareFixedFont{\ttm}{T1}{txtt}{m}{n}{12}  % for normal

% Custom colors
%\usepackage{color}
\definecolor{deepblue}{rgb}{0,0,0.5}
\definecolor{deepred}{rgb}{0.6,0,0}
\definecolor{deepgreen}{rgb}{0,0.5,0}

\usepackage{listings}

% Python style for highlighting
\newcommand\pythonstyle{\lstset{backgroundcolor = \color{lightgray},
language=Python,
basicstyle=\ttm,
otherkeywords={self},             % Add keywords here
keywordstyle=\ttb\color{deepblue},
emph={MyClass,__init__},          % Custom highlighting
emphstyle=\ttb\color{deepred},    % Custom highlighting style
stringstyle=\color{deepgreen},
frame=tb,                         % Any extra options here
showstringspaces=false,            % 
numbers = left
}}


% Python environment
\lstnewenvironment{python}[1][]
{
\pythonstyle
\lstset{#1}
}
{}

% Python for external files
\newcommand\pyExternal[2][]{
\vspace{0.5cm}{
\pythonstyle
\lstinputlisting[#1]{#2}
}
\vspace{0.5cm}}

% Python for inline
\newcommand\pyInline[1]{{\pythonstyle\lstinline!#1!}}



\newcommand\tab[1][1cm]{\hspace*{#1}}

%\usepackage{}



\begin{document}


\section{liste des alcalins} \label{sec:alcalin}

\begin{enumerate}
\item Créer une liste \pyInline{alcalin} contenant les chaînes de caractères Li, Na, K et Rb (liste de métal alcalin)et une autre liste \pyInline{numAtom} qui contient respectivement les numéros atomiques de chacun de ces éléments : 3, 11, 19,37. Afficher ces 2 listes.

\item En demandant à l'utilisateur le numéro dans la liste d'un élément (compris entre 1 et 4) afficher le nom de l'élément correspondant et sa masse atomique.

\item Créer les listes \pyInline{alcalin2} contenant Cs et Fr et \pyInline{numAtom2} contenant 55 et 87 qui correspondent aux données pour les métaux alcalins manquants dans les listes précédentes. A partir de ces 2 listes, compléter les listes \pyInline{alcalin} et \pyInline{numAtom}. Vérifier en affichant \pyInline{alcalin} et \pyInline{numAtom} et tester en affichant le nom du sixième élément  et sa masse atomique.

\item Sans compter le nombre d'éléments, afficher le dernier élément de la liste \pyInline{alcalin}

\item Afficher tous les éléments de la liste sauf le premier

\item Afficher le nombre de métaux alcalins en utilisant 2 fonctions différentes.

\item Afficher tous les éléments de la liste sauf les deux derniers

\item Afficher tous les éléments de la liste à l'envers

\item En utilisant une liste de listes, écrire une liste unique \pyInline{alcalinMetal} qui contient à la fois le symbole de ces métaux et leur numéro atomique.

\item En demandant à l'utilisateur le numéro dans la liste \pyInline{alcalinMetal} d'un élément (compris entre 1 et 4) afficher le nom de l'élément correspondant et sa masse atomique en utilisant \pyInline{alcalinMetal}.

\end{enumerate}



\section{génération de listes d'entiers}

\begin{enumerate}
\item Générer une liste d'entiers allant de 0 à 10 compris

\item Générer une liste d'entiers allant de 0 à 10 compris avec seulement les éléments pairs.

\item Générer une liste d'entiers allant de 5 à 20 compris

\item Générer une liste d'entiers allant de 5 à -10 compris

\end{enumerate}


\section{Remplir une liste avec des éléments saisis par l'utilisateur}

\begin{enumerate}

\item En utilisant une structure \pyInline{for}, demander à l'utilisateur de saisir les différents métaux alcalins (un par un) que vous placerez dans une liste  \pyInline{alcalin} et les numéros atomiques correspondant dans une liste \pyInline{numAtom}. Cette dernière liste devra contenir des entiers. Les différents métaux alcalins sont donnés dans l'exercice \ref{sec:alcalin}

\textbf{Piste} : Vous partirez de listes vides que vous remplirez au fur et à mesure en utilisant \pyInline{append}.

\item Après la saisie de l'ensemble des éléments, afficher sous la forme :

élément : Li; numéro atomique : 3


élément : Na; numéro atomique :11

et ainsi de suite ...

\end{enumerate}


\section{Dosage des ions Calcium et Magnesium dans de l'eau}

Des mesures par dosage de la concentration en ions Calcium et Magnesium sont réalisées. A partir d'une série de mesures, vous allez calculer des grandeurs caractéristiques de cette série.

Afin de générer artificiellement cette série de mesures, vous téléchargerez depuis moodle le fichier \pyInline{giveConcentration.py}. Vous écrirez en début de votre programme les lignes suivantes qui appellent la fonction \pyInline{genereC} du module \pyInline{giveConcentration} pour créer une liste appelée \pyInline{C} de \pyInline{N} concentrations mesurées.

\pyExternal{exempleGiveConcentration.py}

\textbf{Dans cet exercice, vous n'utiliserez pas le module \pyInline{pylab}.}

\begin{enumerate}
\item Calculer la moyenne de la concentration : $\bar{c}$.

\item Trouver le minimum et le maximum de la concentration

\item Calculer l'écart-type de la série de mesure. On donne la formule permettant le calcul de l'écart-type
$$ \sigma=\sqrt{\dfrac{1}{N}\sum_{i=0}^{N-1}(c_i-\bar{c})^2}$$

où $N$ est le nombre de valeurs de la série générée par  \pyInline{genereC}. Les $c_i$ correspondent aux éléments de la matrice des concentrations \pyInline{C} générée par le code ci-dessus.
\end{enumerate}








\end{document}
