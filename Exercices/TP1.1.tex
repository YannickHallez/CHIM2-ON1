
\documentclass[12pt,a4paper]{report}
\usepackage[utf8]{inputenc}
%\usepackage[frenchb]{babel}
\usepackage[T1]{fontenc}
\usepackage{amsmath}
\usepackage{amsfonts}
\usepackage{amssymb}
\usepackage{graphicx}
\usepackage{float} % pour imposer la position avec H
\usepackage{hyperref}
\usepackage{listings} % pour insérer des listings
\usepackage[lofdepth,lotdepth]{subfig} % pôur les sous-figures
\usepackage[french]{isodate}
\usepackage{lmodern}
\usepackage{textcomp}
\usepackage[margin=2cm]{geometry}
\usepackage{tabularx}
\usepackage{hyperref}

\usepackage{xcolor}



\renewcommand{\baselinestretch}{1.00} 


\definecolor{light-gray}{gray}{0.9}

% Default fixed font does not support bold face
\DeclareFixedFont{\ttb}{T1}{txtt}{bx}{n}{12} % for bold
\DeclareFixedFont{\ttm}{T1}{txtt}{m}{n}{12}  % for normal

% Custom colors
%\usepackage{color}
\definecolor{deepblue}{rgb}{0,0,0.5}
\definecolor{deepred}{rgb}{0.6,0,0}
\definecolor{deepgreen}{rgb}{0,0.5,0}

\usepackage{listings}

% Python style for highlighting
\newcommand\pythonstyle{\lstset{backgroundcolor = \color{lightgray},
language=Python,
basicstyle=\ttm,
otherkeywords={self},             % Add keywords here
keywordstyle=\ttb\color{deepblue},
emph={MyClass,__init__},          % Custom highlighting
emphstyle=\ttb\color{deepred},    % Custom highlighting style
stringstyle=\color{deepgreen},
frame=tb,                         % Any extra options here
showstringspaces=false,            % 
numbers = left
}}


% Python environment
\lstnewenvironment{python}[1][]
{
\pythonstyle
\lstset{#1}
}
{}

% Python for external files
\newcommand\pyExternal[2][]{
\vspace{0.5cm}{
\pythonstyle
\lstinputlisting[#1]{#2}
}
\vspace{0.5cm}}

% Python for inline
\newcommand\pyInline[1]{{\pythonstyle\lstinline!#1!}}



\newcommand\tab[1][1cm]{\hspace*{#1}}

%\usepackage{}



\begin{document}





\section{Dosage des ions Calcium et Magnesium dans de l'eau} **

Des mesures par dosage de la concentration en ions Calcium et Magnesium sont réalisées. A partir d'une série de mesures, vous allez calculer des grandeurs caractéristiques de cette série.

Afin de générer artificiellement cette série de mesures, vous téléchargerez depuis moodle le fichier \pyInline{giveConcentration.py}. Vous écrirez en début de votre programme les lignes suivantes qui appellent la fonction \pyInline{genereC} du module \pyInline{giveConcentration} pour créer une \textbf{liste} appelée \pyInline{C} de \pyInline{nbr} concentrations mesurées.

\pyExternal{exempleGiveConcentration.py}

\textbf{Dans cet exercice, vous ne devez pas utiliser de boucles.}

\begin{enumerate}

\item Convertir la liste \pyInline{C} en un tableau.

\item Calculer la moyenne de la concentration : $\bar{c}$.

\item Trouver le minimum et le maximum de la concentration

\item Calculer l'écart-type de la série de mesure sans faire de boucles. On donne la formule permettant le calcul de l'écart-type
$$ \sigma=\sqrt{\dfrac{1}{N-1}\sum_{i=0}^{N}(c_i-\bar{c})^2}$$

où $N$ est le nombre de valeurs de la série générée par  \pyInline{genereC}. Les $c_i$ correspondent aux éléments de la matrice des concentrations \pyInline{C} générée par le code ci-dessus.
\end{enumerate}


\section{écrêtage et lissage de courbes} ***

Des mesures donnent l'évolution d'une concentration en fonction du temps.

Afin de générer artificiellement cette série de mesures, vous téléchargerez depuis moodle le fichier \pyInline{decConcentration.py}. Vous écrirez en début de votre programme les lignes suivantes qui appellent la fonction \pyInline{genereC} du module \pyInline{decConcentration} pour créer une \textbf{liste} appelée \pyInline{C}.

\pyExternal{exempleDecConcentration.py}



\begin{enumerate}
\item Convertir la liste \pyInline{C} en un tableau.

\item Certaines mesures de concentration sont négatives suite à des erreurs de mesure. Sans faire de boucles, remplacer par la valeur 0 les valeurs négatives dans le tableau C.

\item  Les données présentent beaucoup de variations d'une mesure à l'autre. Vous allez lisser ces valeurs en utilisant la méthode de la moyenne mobile. Cette méthode consiste à calculer la ième valeur du tableau des valeurs lissées, que nous appellerons $L$, comme étant la moyenne centrées autour de $C_i$ des k valeurs qui le précèdent, des k valeurs qui le suivent et de lui-même. On a ainsi la formule,

$$L_i=\dfrac{1}{2k+1}\sum_{j=i-k}^{i+k}C_i$$

Calculer les valeurs du tableau $L$. Vous laisserez les k premières valeurs de $L_i$ égales à celle de $C_i$ et de même pour les k dernières.

Vous pouvez en exécutant \pyInline{plot(L)} voir l'effet de votre lissage. Vous pouvez vous amuser à le faire pour différentes valeurs de k.

\item (facultatif) Trouver un moyen d'améliorer le traitement des k premières et des k dernières valeurs pour que le lissage concernent également cette partie des valeurs.
\end{enumerate}








\end{document}
